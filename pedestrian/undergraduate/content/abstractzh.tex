%!TEX root = ../csuthesis_main.tex
% 设置中文摘要
\keywordscn{行人导航\quad 强化学习\quad 虚拟仿真\quad Carla平台\quad 路径规划\quad 避障}
%\categorycn{TP391}
\begin{abstractzh}

交通管理中行人导航关键技术的研究在智慧城市交通系统智能化的背景下,有着重要的意义。基于规则模型或是简单路径规划的方法显然无法适应交通管理的新需求,强化学习作为自适应学习最优策略的方法,具有在动态环境下决策的优势。本文设计基于虚拟引擎和carla平台的行人导航系统,利用强化学习进行路径规划和避障规划,以实现提高避障能力和路径规划效率的目的,在一定程度上能够拓展强化学习在智能交通系统中的应用范畴,并为智慧城市的发展提供技术参考。在快速城市化发展的当下,交通系统智能化的方向关键在于效率与安全,行人导航相关技术研究与应用虽然满足了交通管理的基本需求,但面对更加复杂的动态环境,研发适应性更强的行人导航系统极为必要。在不确定和动态环境下进行决策,强化学习具有较好的应对能力。本文在虚拟仿真技术的基础上进行行人导航系统创新设计,基于虚幻引擎和carla平台搭建了虚拟导航环境,可实时优化路径规划方案和避障策略。之后在多个系统仿真场景进行了有效性实验,诸如导航精度、避障能力、路径规划效率评估等等,验证了系统所基于强化学习的导航系统对于系统动态环境仍具有较为优秀的性能,提高了导航感受和交通效率。研究成果在强化学习在智能交通领域的应用研究上为智慧城市建设提供技术支撑,随着技术的进步和交通需求量的增加,基于强化学习的行人导航系统在城市交通领域将会得到普遍应用。

\end{abstractzh}
