%!TEX root = ../csuthesis_main.tex
\keywordsen{Multi-Target Multi-Camera Tracking (MTMCT)\ \ Person Re-Identification (Re-ID)\ \ Ground Truth\ \ Feature Extraction\ \ Visual Presentation\ \ Detection and Tracking Model Optimization}
\begin{abstracten}

On this basis, we have optimized the existing detection and tracking models. By adjusting and improving the model's parameters and structure, we achieved more than a 0.05 performance increase in at least three out of ten key performance indicators compared to the baseline model. These indicators include, but are not limited to, tracking accuracy, tracking stability, and computational efficiency. Our optimization strategy not only enhances the model's tracking capabilities but also strengthens its robustness in complex traffic scenarios.

To more intuitively demonstrate the performance of our design and existing models, we employed various visualization methods. These methods include heatmaps, trajectory charts, and confusion matrices, which helped us compare the performance of different models in tracking and recognition tasks visually. The visualization results clearly show the advantages of our model, especially in tracking multiple targets and performing person re-identification from different camera perspectives.

Additionally, our research involves the development of an adaptive weighted triplet loss function and the application of hard example mining techniques. The introduction of these technologies further improves the model's feature extraction capabilities in complex scenarios, thereby surpassing tracking performance in the DukeMTMC benchmark and enhancing Re-ID performance in the Market-1501 and DukeMTMC-ReID benchmarks.

Finally, our research not only provides a theoretical in-depth understanding of feature extraction for MTMCT and Re-ID but also offers an effective method for model optimization and performance evaluation in practice. We believe that these achievements will provide strong technical support for the design of future traffic monitoring and safety systems.



\end{abstracten}