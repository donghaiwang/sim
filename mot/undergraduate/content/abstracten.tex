%!TEX root = ../csuthesis_main.tex
\keywordsen{Multi-Target Multi-Camera Tracking (MTMCT)    \ \ Vehicle trajectory reproducibility    \ \ Detection and tracking model optimization    \ \ Multi - intersection vehicle trajectory stitching    \ \ CARLA simulation platform    \ \ Target re - identification (Re - ID) network}
\begin{abstracten}

As smart transportation advances, the demand for precise vehicle tracking technology is rising. In real - world traffic scenarios, conventional single - target tracking struggles to handle multiple targets simultaneously due to issues like target loss in complex scenes, inaccurate multi - sensor data fusion, poor model robustness, and low real - time performance.

Thus, to address the aforementioned issues.This project was completed based on CarlaUE4, PyCharm and Matlab2024b. By integrating CARLA with multi - sensor fusion (LiDAR, camera) and deep learning, it achieved multi - object tracking and evaluation for smart traffic scenarios. 

In CARLA's town10 map, it optimized the detection and tracking model, improving at least 3 out of 10 key indicators (like tracking accuracy, stability, and computational efficiency) by over 5\% compared to the Baseline. This enhanced the model's tracking ability and robustness in complex traffic.

For better presentation, Matlab2024b was used for vehicle trajectory tracking visualization, and PyCharm for trajectory reproducibility visualization. Running the code scripts displays windows showing the vehicle tracking and reproducibility performance, clearly presenting the model's performance in tracking, detection, and trajectory reproducibility, especially in multi - target tracking and vehicle re - identification across different camera views.

Overall, this project deepens the understanding of multi - target multi - camera tracking (MTMCT) and target re - identification (Re - ID) feature extraction. It also offers an effective approach for model optimization and performance evaluation. The findings can strongly support future intelligent traffic monitoring, autonomous driving testing, and digital twin cities.



\end{abstracten}
