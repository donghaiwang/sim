%!TEX root = ../../csuthesis_main.tex

\chapter{总结与展望}

\section{总结}

近年来,自动驾驶技术成为科技界日益关注的热点,各大高校、汽车厂商以及众多互联网公司都在开展自动驾驶算法的研究。据相关数据显示,自动驾驶汽车已在全球多个国家开展测试,并在无人驾驶公交车、无人驾驶出租车、无人驾驶卡车等领域进入商业化阶段。人工智能也正在快速发展,深度学习和强化学习在各自领域展现出强大的威力,取得了令人瞩目的成果。目前市面上的自动驾驶算法研究主要集中在感知系统的开发,在决策和控制方面的进展较为缓慢。而深度强化学习的研究则展现了其在自动驾驶场景中强大的决策能力。因此,本研究希望将深度学习强大的感知能力与传统的车辆控制算法相结合,利用深度强化学习技术进行精准控制,从而改进现有的自动驾驶控制策略。

本文首先介绍了该算法的相关理论框架。深度强化学习基于强化学习的思想,并融合了深度神经网络强大的表征能力,使得强化学习算法能够随着时间推移解决复杂的问题。本文选取与所开发算法相关的部分,例如DQN、PPO算法,进行详细讲解,以帮助读者更好地理解算法设计。

其次,本研究提出基于车路云协同的高精度地图纯电动汽车感知作为自动驾驶汽车的感知层。该方法在高精度静态地图的基础上,通过车云协同、路云协同两种策略,创建能够为车辆提供精准实时道路使用者状况信息的动态图层。目前,大多数基于强化学习的自动驾驶算法使用来自多个摄像头的图像或激光雷达点云作为输入。但由于环境状态空间较大、模型收敛速度较慢,这些方法一般效果较差。本研究选取BEV作为高精度地图算法模型的输入数据,大大提高了认知层的信息密度。

本篇论文研究提出了基于 DQN、PPO 的端到端自主控制算法,此模型借助深度强化学习,可借助高精度地图来学习汽车的决策指南,转化为驾驶指令,这些算法能让车辆在复杂道路上安全行驶,且遵循特定地点的交通规则,论文首先详尽介绍了算法的状态空间、动作空间以及控制层的设计。之后参照分层强化学习理论,针对自主控制任务的各个阶段设计了不同的任务目标以及相应的奖励函数。

文章末尾可看到,本文借助CARLA模拟器构建了CarlaEnv强化学习环境,针对算法的学习阶段与测试阶段设计了两个对比实验,用以证实本文所开发算法有有效性与优良性,本篇论文开发了多种控制场景,运用的自主控制算法以及对照组的其他算法同时开展训练与测试,实验结果显示,本篇论文提出的算法在学习阶段可获取更好的学习成效以及更快的收敛速率。在测试阶段,的算法在给定道路上可良好运行,并且量化性能可接近甚至超越人类手动控制,呈现出算法的智能特性。

\section{展望}

经研究可发现,本篇论文所提出的自动驾驶算法在结构化道路测试里呈现出了一定的智能决策能力,然而在系统完备性以及工程适用性方面依然存在一些有待突破的技术瓶颈,鉴于当前研究存在局限,本课题未来的演进路径可系统性地规划为以下三个核心方向:

(1)多模态交通场景的泛化能力提升

经研究发现,当下多数自动驾驶强化学习算法是依据理想化的道路交通地形结构开展训练的,在场景泛化能力上有着较为十分突出的欠缺,未来研究将聚焦于多级视图传播系统,空间层面,打算整合多模态传感器数据,像LiDAR点云和语义视觉分割等,来构建一个涉及多样化场景的数据库,其中包含恶劣天气状况、突发事故情形、独特道路条件等。时间方面,凭借构建动态生成以分配溢出,并且对车辆、道路、人员和物体的时空互联进行建模,可看到复杂的数字孪生分布机制,比如早晚高峰分布以及重大事件疏散,在交通标志翻译分析工作里,希望能够运用图上的合成神经网络进行高精度地图的拓扑开发、动态再生虚拟交通标志系统,以及在有物理约束的游戏中设计多智能体奖励机制。

(2)奖励函数的自主进化机制研究

研究结果显示,当下奖励函数的设计依旧依赖基于专家经验的特征工程,存在表示不完整以及与环境相关偏差的状况,接下来,会考虑基于深度创造模型的自优化奖励函数结构,首先要构建分层奖励系统,把初始驾驶行为分解成速度合规性、碰撞风险值以及能源效率等量化子目标,借助逆强化学习从人类驾驶数据里学习隐式奖励规则。其次开发一种动态权重分配算法,利用元学习依据学习过程的进展调整奖励系数,来解决多目标优化里的梯度碰撞问题,同时引入因果机制,建立行动和长期收益之间的关联模型,防止因局部最优产生风险策略,另外计划集成联邦学习系统,依靠不同场景下的并行学习达成奖励函数的迁移学习,以此提升算法在不同区域的适配性能。

(3)智能化算法开发平台的生态构建

我们通过研究可以发现,虽然目前的测试方法已经具备了大规模研究的能力,但是硬件和软件之间仍然存在差异。我们计划旨在打造“一站式”智能研发生态:在架构最底层,利用容器化技术打造去中心化的学习课堂,配合ONNX架构改造和TensorRT加速,实现应用性能的大幅提升;在中层,开发了低代码可视化设计框架,提供场景生成(如道路分割、产品分割)、通过“拖拽”的方式组装算法模块(强化学习/模拟学习三步流程/热初始化流程)等基础任务。在网络应用的顶层,自主访问控制(SANB)系统设计为支持按照GB/T 38186-2021标准进行通信,包括基于ISO 26262的主动安全分析模块,以及与RARLA等关键架构进行通信的开放API。通过开发完整的“数据训练-验证-处理”工具,将算法的计算时间降低到目前的百分之三十以下。

