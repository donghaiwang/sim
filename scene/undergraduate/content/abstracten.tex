%!TEX root = ../csuthesis_main.tex
\keywordsen{Intelligent Driving\ \ Test Scenarios\ \ Large Language Models\ \ Scenario Description Language\ \ Scenic}
\begin{abstracten}
	
	The low efficiency of constructing test scenarios for intelligent driving is a significant issue faced by the industry. To address this, this paper proposes a novel generation framework that integrates Large Language Models (LLMs) with formal scenario description languages, aiming to efficiently transform natural language descriptions into high-fidelity 3D test scenarios. The framework consists of three core modules: First, a domain knowledge-enhanced instruction parsing module that accurately understands the test scenario requirements described in natural language, laying the foundation for subsequent generation tasks; second, a syntax-semantic dual-validation code generation mechanism that ensures the generated Scenic code is not only syntactically correct but also semantically logical, thereby improving the usability and accuracy of the code; and third, a physics rule-driven scenario synthesis engine that reasonably generates 3D scenarios based on physical rules, ensuring the realism and reliability of the scenarios. In experimental validation on the CARLA simulation platform, the system demonstrated excellent performance: the accuracy of the generated Scenic code reached 92.3%, the physical compliance rate of the scenarios exceeded 85%, and the generation latency was less than 3 seconds per scenario. This achievement significantly improves the efficiency of constructing test scenarios for intelligent driving, providing an efficient and accurate tool for the testing and optimization of intelligent driving technologies. It is expected to promote the efficient conduct of related testing work in the intelligent driving field and advance the further development and improvement of intelligent driving technologies, with significant application value and broad development prospects.
	
\end{abstracten}