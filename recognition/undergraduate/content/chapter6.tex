%!TEX root = ../../csuthesis_main.tex
\chapter{总结与展望}

\section{研究总结}

本研究围绕类脑视觉建模与评估展开,重点探讨了基于CORnet模型的复现、优化与神经对齐能力分析。首先,在Tiny-ImageNet数据集上成功复现了CORnet-S模型,实验结果显示该模型具备较强的图像识别能力,验证集和测试集的 Top-1 准确率分别达到了60.34\%和63.67\%,性能接近原论文所述水平。

在此基础上,本文以CORnet-Z模型为框架,引入通道注意力机制(Squeeze-and-Excitation,SE)以提升模型对判别性特征的感知能力。通过对比实验发现,引入SE模块后模型在验证集和测试集上的Top-1与Top-5准确率均有所提升,说明注意力机制在一定程度上改善了特征提取效果,提升了模型在复杂图像场景中的识别性能。

此外,本文还基于Brain-Score提供的MajajHong2015数据集,对原始CORnet-Z、改进后的CORnet-Z+SE以及其他模型(如ResNet-18、AlexNet、CBAM版本、VOneBlock版本)进行了类脑相似性评估。从结果来看,原始CORnet-Z模型在IT层的神经预测得分较高,而加入SE、CBAM或VOneBlock等结构后,虽然提升了图像分类准确率,但在类脑得分方面出现了一定程度的下降,表明部分优化方式在类脑性和准确性之间存在权衡。

为了进一步理解模型的内部机制,本文还对CORnet-Z及其优化版本在各层的激活图进行了可视化分析。结果显示,注意力机制引导模型对局部目标区域(如车轮、车身)产生更集中的响应,表现出较强的局部结构识别能力。但同时也可能削弱了模型对整体空间结构的编码能力,这在一定程度上解释了类脑相似性指标下降的原因。

\section{未来展望}

尽管本研究在模型复现、结构优化以及类脑评估方面已取得一定进展,但仍有多个方向值得进一步深入探索。

目前所采用的CORnet系列模型主要基于前馈结构,在信息流动的动态性和反馈调节机制方面仍显不足。未来的模型设计可以考虑引入跨层反馈连接,用以模拟大脑皮层中V1至IT区域之间的再入式信息交互,从而提升模型对复杂视觉场景中时间动态的建模能力和整体稳健性。

另一个值得关注的问题是识别准确率与类脑结构相似性之间的平衡。在实际建模过程中,优化识别性能往往会牺牲部分神经一致性,因此如何在保持高分类能力的同时增强模型的神经可对齐性,是未来构建高质量类脑模型的关键挑战。可以尝试从稀疏编码、生物解剖结构约束等角度出发,设计兼顾表现力与生物合理性的混合模型架构。

类脑评估体系的完善也是后续研究的重要方向。目前Brain-Score所依赖的神经数据多来源于静态中心注视图像和单一任务条件,难以覆盖真实环境下的复杂动态场景。在未来的评估体系中,可考虑引入视频理解、跨模态匹配等任务类型,同时结合fMRI、MEG等多源脑成像数据,拓展模型对人脑处理机制的拟合深度和广度。

在工程部署层面,提升模型对多种硬件平台的适配能力也具有实际意义。当前类脑模型大多面向通用GPU平台构建,若能进一步探索其在神经形态芯片(如Intel Loihi、寒武纪MLU)上的部署策略,有望在保持类脑结构特性的同时实现更高效的边缘计算与实时响应。

总体来看,类脑视觉模型作为人工智能与神经科学交叉领域的前沿方向,仍处于持续发展的早期阶段。本文的研究工作虽然规模有限,但在模型结构改进与评估实践方面提供了可供参考的思路,为后续深入探索类脑机制与视觉智能融合奠定了一定基础。

