%!TEX root = ../../csuthesis_main.tex

\chapter{绪论}

\section{研究背景及意义}

\subsection{研究背景}

随着深度学习的快速发展,卷积神经网络(Convolutional Neural Networks, CNN)在计算机视觉任务中取得了显著成果,在图像分类、目标检测和语义分割等领域已接近或超越人类水平。然而,尽管这些模型在性能指标上表现优异,其内部计算机制与人类大脑视觉处理过程仍存在显著差异。这种“黑箱式”的建模方式在可解释性、泛化能力以及对抗鲁棒性方面仍面临诸多挑战。近年来,神经科学领域的研究逐渐揭示出人类视觉系统在处理信息时表现出的层级结构、时间动态性与注意力调节机制,这些发现为人工智能模型的类脑设计提供了理论基础。

为进一步提升视觉模型的生物合理性,研究者们提出了类脑视觉模型的概念,试图将神经科学中有关视觉皮层的结构特征与深度网络架构相结合。其中,CORnet家族模型是一类典型的类脑模型,其通过模拟大脑腹侧视觉通路(V1-V2-V4-IT)层级结构,结合时间反馈和非线性变换机制,在保留网络可训练性的同时,引入了生物启发性的模块设计。与此同时,Brain-Score作为一种衡量神经网络类脑程度的综合性评估体系,也逐渐成为类脑视觉模型验证的重要工具。

\subsection{研究目的}

本文旨在探讨并优化CORnet系列模型在图像识别任务中的表现,并讨论不同模型在人脑视觉皮层响应模式的一致性。为此,本文首先复现了CORnet-S模型,并在Tiny-ImageNet-200数据集上验证其基线性能;随后以CORnet-Z模型为基础,设计并引入通道注意力机制(SE模块),从结构上增强模型的特征选择能力;最后,使用Brain-Score框架评估不同模型在V4和IT脑区的类脑相似性得分,并通过激活图可视化进一步分析不同模型对图像区域的关注模式差异。

\subsection{研究意义}

本研究具有一定的的理论价值与实际应用意义。在理论层面,本文以类脑视觉模型CORnet-Z为基础,尝试引入通道注意力机制,以探讨注意力机制在类脑模型结构中的作用。通过使用Brain-Score指标,能够从神经表征的角度,对模型与人脑视觉皮层在不同层级上的相似性进行量化评估,为深入理解人工模型与生物视觉系统的关系提供了参考。在实践层面,引入SE模块的模型在图像分类任务中取得了小幅准确率提升,说明合理的结构改进有助于提升模型的识别性能。同时,类脑相似性评估结果也表明,结构优化不一定带来更好的类脑表现,这对于今后的模型设计提供了启示。此外,本文还通过可视化分析比较了模型各层的激活特征,为理解模型的内部处理机制提供了直观支持,也提升了模型的可解释性。

\section{国内外研究现状}

\subsection{国外研究进展}

近年来,国外关于类脑视觉模型的研究取得了较为显著的进展。特别是在认知神经科学与人工智能交叉领域,研究者们试图构建更贴近人脑处理机制的视觉识别模型。Yamins等人(2014)基于HMO架构,提出使用深度CNN预测IT神经元活动,模型得分与行为数据显著相关,被视为第一个类脑预测型网络。一年后MIT的DiCarlo课题组早期通过电生理记录提出IT皮层对高层视觉语义的编码规律(Majaj et al.,2015),并据此构建92幅HVM标准刺激集,成为后续神经网络对齐研究的重要基准。而后Schrimpf等人提出的Brain-Score评估体系(Schrimpf et al.,2018)为神经网络与大脑神经响应之间的一致性提供了量化工具,推动了类脑模型的系统评估。

在模型架构设计方面,Yamins等人早期提出的Hierarchical Modular Network(HMO)模型已被证明能够较好地预测灵长类动物IT皮层的神经活动(Yamins et al.,2014),为后续类脑建模奠定了基础。随后,Kubilius等人构建了CORnet模型家族(Kubilius et al.,2019),其中CORnet-Z是最简单的版本,CORnet-S则映入时间递归链接模仿生物大脑的动态反馈特性成为最具代表性的类脑模型。它们在结构设计上模仿人脑视觉通路中的V1、V2、V4和IT四个主要区域,并在多个神经一致性基准上取得了较好成绩。

除此之外,一些研究还关注将注意力机制与类脑模型结合。如Jetley等人在2018年提出了基于通道注意力机制的图像分类网络(Jetley et al.,2018)。同一年,Hu等人提出Squeeze-and-Excitation网络,利用全局平均池化建模通道之间的依赖关系,在ImageNet上提高2\%以上的Top-1准确率。Woo等(2018)设计CBAM模块,引入空间注意力对显著区域进行强化,在目标检测任务中同样表现优异。尽管这些模块并非为类脑模型设计,但它们的生物启发式结构为后面的研究提供了重要的工具基础。

\subsection{国内研究趋势与特色}

相较于国外起步较早、资源丰富,国内在类脑视觉建模方面起步相对较晚,但近五年来发展十分迅速,并形成了多条极具中国特色的研究路径。在模型设计方面,很多研究机构都将“类脑计算”纳入了重点的研究方向。例如中国科学院自动化研究所开发的BrainCog平台,这个平台集成了很多种神经建模工具,推动了类脑建模的标准化发展。清华大学和浙江大学等一流高校在视觉神经计算与模式识别方面也已发表多篇相关论文,尝试将神经生理机制引入深度神经网络中。

在2018年北京大学黄铁军团队设计出模仿视网膜采样机制的事件驱动图像芯片,通过模拟生物视觉的稀疏编码特性,在低功耗条件下实现高效图像处理。该芯片与类脑CNN结合时,单位功耗处理能力提升达百倍,推动“类脑感知”向实际应用落地。中国科学技术大学王田苗团队(2023)尝试基于ERP脑电数据对网络中间层特征进行同步比对,初步构建视觉加工时间窗与网络响应时序之间的对映关系。虽尚未达到脑区级精度,但在类脑模型动态行为评估方面提供了新的视角。清华大学李华等(2025)提出ReAlNet-fMRI框架,首次将人类fMRI数据引入CNN训练流程,通过设计层级响应对齐损失项,使模型在多个皮层区域的神经预测准确率显著提升,验证类脑模型结构可通过神经数据进行“监督调整”。

\subsection{现有研究不足}

当前类脑视觉模型在跨层级反馈机制构建上任然存在着明显的不足。生物视觉系统依赖腹侧通路与背侧通路的协同,尤其是顶叶皮层LIP区至初级视皮层V1区的反馈连接,对注意力分配至关重要。主流模型如CORnet系列仍采用单向前馈架构,仅模拟腹侧通路的时序传递,跨脑区的反馈机制还没有被有效整合。这种简化导致模型在处理动态遮挡、视角变化等复杂任务时,缺乏生物视觉系统的特征重校准能力,与神经解剖结构存在本质差距。

现有评估体系在场景多样性和任务复杂度上显得单一。研究主要依赖ImageNet等静态数据集,虽然Brain-Score能够量化模型神经响应与灵长类记录的相似性,但评估场景局限于实验室标准化刺激。这难以反映模型在真实世界的动态感知能力,如光照突变、运动模糊等条件下的鲁棒性。现有指标缺乏对任务驱动型注意力机制的量化评估,无法衡量模型在主动视觉任务中的生物合理性,导致评估结果与实际应用需求脱节。

硬件协同设计的滞后也制约了类脑视觉模型的实用化。研究多集中于算法优化,而忽视神经形态芯片、类脑传感器等硬件的开发。轻量化模型如CORnet-Z虽具实时处理潜力,但在传统GPU上能效比低。国外在神经形态芯片方向已有初步探索,例如IBM TrueNorth和Intel Loihi,但与现有模型的密集计算需求适配困难。国内在类脑计算芯片上虽然取得了突破,例如寒武纪MLU系列,但在脉冲神经网络与深度学习模型的异构融合上尚未成熟,导致算法创新难以转化为实际应用效能。

\section{本文研究内容与结构}

\subsection{本文研究内容}

本文围绕类脑视觉模型CORnet-Z展开,结合神经科学启发与模型结构改进,对其识别性能与类脑相似性进行了系统研究。主要工作如下:

\subsubsection{复现CORnet-S模型,并在Tiny-ImageNet-200数据集上进行训练与评估,获得验证集与测试集的Top-1和Top-5准确率,用以作为基线参考,验证模型在轻量级数据集上的表现稳定性;}

\subsubsection{在CORnet-Z模型基础上引入通道注意力机制(Squeeze-and-Excitation模块),构建改进模型CORnet-Z+SE,旨在增强模型对关键信息通道的选择性表达能力,并观察其对分类准确率的影响;}

\subsubsection{利用Brain-Score评估框架,对原始CORnet-Z、引入SE模块的CORnet-Z、进一步引入CBAM注意力机制与VOneBlock的结构改进模型进行神经一致性评估,分析各模型在V4与IT层的类脑得分变化趋势,并与ResNet-18、AlexNet等典型模型进行横向比较;}

\subsubsection{基于Grad-CAM可视化方法,提取并对比CORnet-Z模型与改进后模型在不同层级的激活图,观察注意力机制对特征图关注区域的调节作用,并初步探讨模型内部的可解释性变化。}

通过上述工作,本文试图探讨注意力机制在类脑视觉模型中的作用效果,以及结构改动对识别准确率与神经一致性之间可能存在的权衡关系。

\subsection{本文结构安排}

全文共分为六章,各章内容如下:

第一章为绪论,介绍课题背景、研究意义,综述国内外研究现状,并明确本文的研究目标与内容框架。

第二章介绍类脑视觉模型的理论基础,包括人类视觉系统的层级结构、反馈机制与注意力调节原理,重点分析CORnet模型家族的设计思路以及Brain-Score等评估标准的构成方式。

第三章描述CORnet-S模型的复现过程与性能评估,列举实验配置与结果,作为后续结构优化的参考基线。

第四章针对CORnet-Z模型展开结构优化,重点介绍SE注意力机制的设计与集成方法,分析改进后模型在分类准确率方面的表现。

第五章开展类脑相似性评估与激活图可视化实验,比较不同模型在神经预测指标上的得分差异,并结合图像激活特征探讨其内部表征的变化规律。

第六章对全文进行总结,归纳研究成果与发现,分析研究中存在的不足,并提出未来可能的改进方向。

