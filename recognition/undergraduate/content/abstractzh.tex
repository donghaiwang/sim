%!TEX root = ../csuthesis_main.tex
% 设置中文摘要
\keywordscn{类脑视觉;CORnet模型;通道注意力;Brain-Score评估;神经响应相似性}
%\categorycn{TP391}
\begin{abstractzh}

类脑视觉模型旨在模拟人类大脑视觉皮层的信息处理方式,在计算机视觉与神经科学交叉研究中具有重要意义。本文以CORnet系列模型为基础,围绕识别性能优化与类脑相似性分析两个方面展开研究。首先,在Tiny-ImageNet数据集上复现了CORnet-S模型,并验证其在轻量化结构下的图像识别能力。随后,在CORnet-Z模型中引入Squeeze-and-Excitation(SE)通道注意力机制,构建了改进模型CORnet-Z+SE,实验结果表明其Top-1和Top-5准确率均有一定提升。

为评估模型的类脑性,本文使用Brain-Score框架,基于MajajHong2015公开神经数据集,对多个模型在V4和IT脑区进行相似性测试。结果显示,引入注意力机制虽提升了分类准确率,但类脑得分下降,说明判别性能与神经对齐之间存在一定权衡关系。通过对各层激活图的可视化分析,进一步观察到SE模块增强了模型对目标局部区域的响应,但可能减少了整体响应的广度,影响其了生物一致性。


\end{abstractzh}
