%!TEX root = ../../csuthesis_main.tex
\chapter{引言}

跟着科技快速地向前发展,智能驾驶技术慢慢从实验室迈向现实道路,成为汽车工业和人工智能领域重要研究方向,智能驾驶系统依靠集成先进的传感器、控制器、执行器以及复杂算法,实现对车辆的自主控制和决策,目的是提高道路安全性、缓解交通拥堵、降低能源消耗并提升驾驶舒适性,不过,智能驾驶系统的安全性和可靠性一直是其发展方面的关键瓶颈,在复杂多变的道路交通环境当中,智能驾驶系统得面对各种突发状况和潜在危险,像行人突然横穿马路、车辆紧急变道、恶劣天气条件等,这些场景对系统的感知、决策和控制能力提出非常高的要求。

\section{研究背景}


为了在智能驾驶系统真正投入实际应用前充分验证其性能,传统道路测试方法存在着诸多明显局限性,在测试周期方面,针对自动驾驶出租车开展道路测试时,因要进行大量实际路况测试与优化,测试周期通常会长达数年之久,这严重延误产品上市时间还增加研发成本,以自动驾驶卡车为例来说,购置及改装测试车辆、支付测试人员薪酬、租赁测试场地等费用,让道路测试成本变得十分高昂,一次大规模道路测试项目成本可能高达数百万甚至上千万元,而且道路测试很难覆盖所有可能出现的危险场景,尤其是那些发生概率较低但后果严重的事故场景,比如在特殊天气条件之下,像暴风雪导致道路结冰、浓雾影响视线等情况,以及罕见的交通冲突场景,例如多车同时失控这种状况,这些场景很难通过道路测试进行全面模拟,而它们对智能驾驶系统安全性存在潜在巨大威胁,此外道路测试还存在一定程度的安全风险,可能会对测试人员和周围环境造成相应威胁,曾经有道路测试中出现车辆失控情况,导致测试人员受伤以及周围建筑物损坏案例,这进一步凸显传统道路测试在安全方面存在的隐患。

所以仿真技术就这么应运而生了,它成了智能驾驶系统测试和验证的重要手段,通过运用先进的计算机图形学与传感器建模技术,能在计算机里构建出高度逼真的虚拟道路交通环境,在这个虚拟环境当中,可以精准模拟不同的天气条件,像晴天时候的强光照射、雨天时候的路面湿滑、夜晚时候的低光照环境等,还有不同的交通流量情况,包含高峰时段的拥堵和复杂交通交互、平峰时段的顺畅行驶等,还能够精确控制测试条件和参数,\cite{dingus2006100}进而实现对智能驾驶系统的全面测试,和道路测试比较起来,一次仿真测试的成本可能仅仅是道路测试的几十分之一,大幅降低了测试成本,而且测试过程完全可控,不存在任何安全风险=。

在仿真场景的构建过程当中,危险场景的生成和优化特别关键,危险场景指的是那些可能让智能驾驶系统出现错误决策、控制失效或者发生事故的场景,准确生成和优化这些危险场景,能够帮助研究人员深入理解智能驾驶系统面对危险时的行为表现,发现系统潜在的缺陷和不足之处,进而有针对性地改进系统的设计和相关算法,以此提高系统的鲁棒性和安全性能,然而危险场景的生成并非是一件容易的事情,需要综合考虑各种各样的因素,道路交通规则在不同国家和地区存在着明显差异,比如有些国家车辆靠左行驶而另一些国家则靠右行驶,这就要求在生成危险场景时充分考虑这些规则差异,车辆动力学特性像车辆的加速、制动、转向性能等会影响危险场景的生成,以快速制动场景来说不同车辆制动距离和制动响应时间不同,这就需要在场景生成中进行准确模拟,传感器感知范围如激光雷达视野角度和精度、摄像头成像质量和识别能力等,同样是生成危险场景时需要重点考虑的因素,环境因素如光照强度和温度会对智能驾驶系统性能产生影响,例如强光照射下传感器可能出现误判、高温环境下车辆电子设备可能出现故障。要保证生成的场景有高度真实性和代表性,就得结合实际道路数据像交通事故数据、交通流量数据等\cite{benmimoun2011incident},并且利用机器学习和深度学习算法来生成能真实反映智能驾驶系统在实际道路可能遇到危险情况的场景,通过对这些危险场景进行分析,能够发现智能驾驶系统在决策逻辑和控制算法方面存在的缺陷,比如在某些复杂交通场景中系统决策速度过慢或者决策错误,进而有针对性地优化决策算法,提升系统对危险场景的识别和应对能力,以此确保智能驾驶系统的安全性和可靠性。



\section{研究目的和意义}

\subsection{研究目的}
本研究旨在深入探讨智能驾驶危险仿真场景的生成和优化技术,以期为智能驾驶系统的安全测试和性能提升提供有力支持。研究目的具体包括:

系统地对现有的智能驾驶危险仿真场景生成方法进行梳理和总结,运用文献计量分析、技术路线对比以及案例实证研究等方式,建立起涵盖基于规则驱动、数据驱动、模型驱动等主流方法的理论图谱。深入剖析各个方法在场景真实性、计算效率、泛化能力等不同维度的优点和缺点,结合ISO 21448预期功能安全标准与UN R157自动驾驶法规要求,明确不同方法在高速换道、交叉口通行、极端天气应对等特定场景下的适用边界。通过构建方法评估指标体系,为后续的算法创新提供量化的参照标准\cite{najm2007pre},同时揭示当前研究在场景动态演化模拟方面的情况,多主体交互建模等领域的理论空白,为学术研究与工程实践指明方向。​

针对现有算法在场景多样性、实时性和准确性方面存在的不足,融合强化学习、生成对抗网络(GAN)与数字孪生技术来研发新一代危险场景生成算法,基于真实交通事故数据库与高精度地图数据,利用迁移学习技术构建场景特征提取模型以实现从历史数据到虚拟场景的高效映射,引入多智能体协同演化机制来模拟车辆、行人\cite{苏江平2017基于中国危险工况的行人交通冲突典型场景提取与分析}、基础设施间的复杂交互行为,生成包含长尾风险场景如鬼探头、道路施工突发状况在内的多样化测试用例,通过GPU并行计算与分布式架构优化将场景生成效率提升30\%以上以满足需求。大规模仿真测试的实时性需求,为智能驾驶系统的压力测试与漏洞挖掘提供技术支撑。​

建立起场景挑战性和测试覆盖率的量化关联模型,针对智能驾驶系统感知、决策、控制这三大核心模块的薄弱环节,设计分层递进式的场景优化方案,在感知层通过调整光照强度、遮挡物分布等参数来测试传感器鲁棒性,在决策层设置冲突优先级、博弈策略等变量评估算法逻辑合理性,在控制层改变路面摩擦系数、车辆动力学参数验证执行机构可靠性,构建场景难度动态评估体系实现测试场景从基础场景到极限场景的自适应,使仿真测试更贴合实际道路风险分布,有效提升系统极端工况下的安全冗余设计能力。​​

按照微服务架构和容器化技术来构建一体化框架,这个框架包含数据采集、场景生成、智能优化、仿真测试、结果评估这五大核心模块,开发标准化的数据接口,以兼容多源异构数据像激光雷达点云、摄像头图像、高精地图矢量数据并实现高效处理,集成主流仿真引擎比如CARLA、PreScan与自动驾驶开发平台如Autoware、Apollo,达成场景快速部署和跨平台测试,引入数字线程技术,打通场景设计、仿真执行、问题反馈的闭环链路,以此支持测试流程自动化编排与迭代优化,最终形成一套具备高扩展性和高复用性的工程化工具,可显著降低智能驾驶系统开发测试成本并加速技术产业化落地进程。

本研究的意义主要体现在以下几个方面:

\subsection{意义}
理论意义:丰富和完善智能驾驶仿真测试领域的理论体系,为智能驾驶危险场景的生成和优化提供新的思路和方法,推动相关学术研究的深入发展。

技术意义:提高智能驾驶仿真测试技术的水平,为智能驾驶系统的开发和测试提供更高效、更可靠的工具,促进智能驾驶技术的成熟和应用。

实际应用意义在于通过生成并优化危险仿真场景,可更全面地评估智能驾驶系统安全性,能提前发现并解决潜在的安全方面问题,进而降低智能驾驶系统在实际应用时的风险,为智能驾驶车辆的上路行驶提供安全保障,具备重要的社会价值与经济价值。

\section{国内外研究现状}

国外研究现状场景生成方法与理论体系:

德国PEGASUS项目产生的影响十分深远,它所定义的“功能,逻辑,具体”三级分层体系,为场景构建提供了清晰的结构化思路,能够从抽象的功能需求逐步细化成具体可执行的仿真场景描述,围绕这个体系构建了面向智能驾驶全生命周期即概念到开发再到测试以及标定的场景库流程和测试方法,为整个行业树立了良好的标杆\cite{hiemann2008estimating},与此同时部分学者从参数赋值的视角切入,运用仿真技术识别关键场景并且建立起车辆安全与交通质量的性能评价指标,让场景评估有了量化的依据,除此之外基于自动编码器自动识别独特场景、利用行驶距离侵入,碰撞时间划分危险区域生成测试案例等方法不断涌现出来,丰富了场景生成的手段与相应策略。

仿真技术与工具应用方面在仿真工具层面德国 VIRES 公司的 VTD 功能十分全面涵盖道路环境建模交通场景搭建等功能支持智能驾驶全周期开发流程且依托开放格式便于与第三方工具联合仿真,PreScan 作为 ADAS 测试仿真软件具备强大场景搭建能力支持丰富传感器类型可与第三方动力学模型和 HIL 模拟器集成应用范围广泛,CarMaker 不仅车辆动力学模型精准还打造闭环仿真系统可与第三方软件集成满足多样化仿真需求,Vissim 专注于微观交通流仿真能高精度模拟复杂交通环境及交互行为为智能驾驶算法测试提供有力支持 。Vissim 专注于微观交通流仿真,能高精度模拟复杂交通环境及多交通参与者交互行为,为智能驾驶算法在复杂交通场景下的测试提供了有力支持 。

安全模型与场景优化方面,在安全模型的研究进程中诸多经典模型持续演进,像针对跟车场景的模型通过假设不同转向轨迹精准计算安全距离与碰撞车速来提前规划制动或转向策略,Safety Zone模型在行人横穿场景里借助扩展虚拟安全空间保障行人安全,RSS模型清晰定义安全距离和路权归属在多车交互场景中优势较为明显,SFF模型从时空一体化角度利用“安全力场”避免轨迹出现重叠情况,FSM模型采用模糊逻辑精细捕捉安全状态的变化,这些模型为危险场景的优化与评估提供了相应理论依据,通过调整场景参数如在感知层改变光照和遮挡、在决策层设置冲突优先级、在控制层调整路面摩擦系数等提升场景对智能驾驶系统极限性能的测试效能。

国内研究现状:

场景构建与标准制定:国内积极紧跟国际步伐开展场景构建与标准制定工作,踊跃参与智能驾驶仿真场景相关标准制定事宜,比如围绕 ISO 21448 预期功能安全标准开展本土化研究并推动落地,部分企业和科研机构参考国内外架构设计方法,结合自身工程实践经验,提出基于关键要素分析的场景搭建框架,该框架涵盖道路、环境、主车、交通参与物四类关键特征,生成的场景既能体现触发条件又便于在模拟仿真软件中搭建,有效提升了自动驾驶系统测试验证的效率\cite{薛薇2009SPSS}。

数据驱动的场景生成方面,随着国内大数据和人工智能技术大力发展,基于真实交通事故数据库以及高精度地图数据,利用迁移学习技术构建场景特征提取模型成研究热点,其目的是从海量数据里高效提取危险场景特征并生成多样化测试用例,同时借助多智能体协同演化机制模拟复杂交通交互行为,增加场景真实性与复杂性,尤其在对“鬼探头”、道路施工突发状况等长尾风险场景的模拟上取得一定进展。

国内在一体化框架与平台研发方面,努力构建智能驾驶危险仿真场景生成和优化的一体化框架,凭借微服务架构与容器化技术,将数据采集、场景生成、优化、仿真测试、结果评估等环节进行整合,开发出标准化数据接口用以兼容多源异构数据处理,集成包括在国内广泛应用的CARLA等主流仿真引擎,以及像百度Apollo、Autoware本土化应用这类自动驾驶开发平台,并且引入数字线程技术来实现测试流程自动化编排与迭代优化\cite{杨学兵2007决策树算法及其核心技术},为智能驾驶系统开发测试提供一站式解决方案。

\section{研究方法}

数据驱动分析:

交通事故数据分析就是去收集并且整理大量交通事故案例数据,这些数据要涵盖不同类型事故像追尾、侧撞、翻车等,同时还要包括事故发生的各种因素如天气条件、道路状况、车辆类型、驾驶行为等,运用数据挖掘技术例如关联规则挖掘、聚类分析等,从海量数据当中挖掘出事故发生的潜在规律和关键影响因素,从而识别出常见的危险场景模式,比如通过分析能够发现,在雨天湿滑路面上车辆高速行驶时容易发生侧滑进而引发侧撞事故,或者在交通拥堵路段车辆频繁变道加塞等行为容易导致追尾事故\cite{曾星2018基于深度传感器的坐姿检测系统}。

驾驶行为数据挖掘:采集大量的驾驶行为数据,涵盖车辆行驶轨迹、速度、加速度、转向角度和刹车力度等方面,还有驾驶员踩油门、踩刹车、打方向盘等操作行为\cite{吴斌2018基于自然驾驶研究的直行追尾危险场景诱导因素分析}。借助随机森林、支持向量机等机器学习算法,对驾驶行为数据开展分类与预测工作,识别危险驾驶行为模式。举例来说,经过分析能够发现,驾驶员在高速行驶时突然急刹车、急转弯这类行为易造成车辆失控,从而引发事故,又或者在跟车过程中车距过近且跟车速度过快,容易出现追尾事故\cite{程娟2019基于梯度提升决策树的高速公路行程时间预测模型}。

危险场景类别定义与参数设置:

场景分类框架构建是基于数据驱动分析结果构建系统化危险场景分类框架,把危险场景按不同维度分类像按事故类型如追尾侧撞翻车等、道路类型如高速公路城市道路乡村道路等、天气条件如晴天雨天雪天等、交通状况如拥堵畅通等,并且为每一类危险场景设定具体参数和条件像车辆速度范围加速度范围转向角度范围车距范围等,比如在高速公路追尾场景中设定车辆初始速度为80 - 120km/h车距为10 - 20米加速度为 - 5 - 0m/s²等\cite{chen2016xgboost}。

仿真环境参数配置方面要使用CARLA等仿真软件,依据所定义的危险场景类别和参数来设置仿真环境里的各类参数,这些参数涵盖天气参数像雨量大小雪量大小雾浓度等情况、光照参数如太阳高度角光照强度等方面、道路参数例如路面材质摩擦系数等内容、交通密度参数诸如车辆数量行人数量等情形,以此来模拟真实世界当中的各种复杂条件 。例如,在模拟雨天追尾场景时,设置雨量为中雨,路面摩擦系数降低20\%,光照强度降低30\%等。

手动控制场景实现:

用Python语言来编写脚本用于控制仿真环境里车辆和行人行为,在脚本里设定车辆初始位置、速度、加速度、转向角度等相关参数,同时设定行人行走路径、速度等这些参数,举例来说可以编写一个脚本让车辆在高速公路某一车道以100km/h速度行驶,之后在某一时刻让其突然急刹车,并且让另一辆车以更高速度从后方驶来以此模拟追尾场景\cite{师圣蔓2019基于机器学习的网络流量预测与应用研究}。

初始状态配置工作要在仿真开始之前完成,需要把场景的初始状态都配置妥当,这里面涵盖车辆的初始位置、速度以及方向等内容,同时也包括行人的初始位置和速度等方面,要保证场景的初始状态能够符合所设定的危险场景类别以及相关参数要求,就像在模拟行人横穿马路这一场景的时候,要把行人初始位置设定在道路一侧的人行道上,将其速度设定为正常步行速度,并且把车辆初始位置设定在距离行人一定距离的道路上,把车辆速度设定为正常行驶速度\cite{靳小波0基于机器学习算法的文本分类系统}。


数据收集与效果量化:

传感器数据采集指的是在仿真过程当中收集车辆传感器数据,涵盖速度、位置、加速度、转向角度以及刹车力度等内容,这些数据能够通过仿真软件所提供的 API 接口来获取,比如在 CARLA 里面可以利用车辆的传感器组件获取车辆实时数据\cite{ke2017lightgbm}。

事件记录方面要记录仿真当中的关键事件,像碰撞、紧急制动以及车道偏离等情况,这些事件能够借助仿真软件的事件监听机制来达成,比如在CARLA这个仿真环境里,可以运用事件监听器对车辆的碰撞事件进行监听,并且记录下碰撞发生的具体时间、所在位置以及当时速度等相关信息。

评价指标设定方面要设定量化仿真效果的评价指标像碰撞率反应时间安全距离保持率等这些指标可用于评估生成的危险场景对自动驾驶系统的影响比如碰撞率。

数据分析就是运用统计方法对收集来的数据做分析,以此评估场景生成出来的效果,比如通过计算碰撞率的平均值以及标准差等统计量,来分析不同危险场景对自动驾驶系统的影响程度,又或者通过绘制反应时间的分布图,去了解自动驾驶系统在不同场景下的反应速度。

结果展示:

通过图表来展示仿真结果,使用像Matplotlib、Seaborn这类数据可视化工具制作相关图表,例如绘制碰撞率随仿真时间变化的折线图,又或者绘制反应时间的直方图,以此直观地呈现仿真结果。

视频展示会进行仿真视频剪辑来直观呈现场景和车辆行为,比如把仿真过程里生成的对抗性场景录制成视频,用以展示车辆于复杂交通环境当中的行驶状况,还有自动驾驶系统在面临危险时的决策与控制过程。

场景实现与验证:

API应用方面是使用CARLA API来开展场景脚本开发工作实现自动化场景生成过程,借助API调用可达成对车辆行人环境等方面的控制并实现对仿真过程监控与数据采集,仿真测试则是在仿真环境里测试场景以验证其合理性和挑战性,通过多次仿真测试来观察场景运行状况检查是否存在不合理或不符合预期的情况,要依据测试结果去调整场景参数优化场景设计,比如在测试行人横穿马路场景时要观察行人能否安全通过马路以及自动驾驶系统能否及时做出避让反应\cite{刘彧祺2019基于} 。

仿真场景构建技术研究:

研究综述部分是对场景自动构建方法开展研究综述工作,会总结目前存在的各类方法,比如基于规则的方法、基于数据驱动的方法以及基于模型的方法等,同时还会分析这些方法的优缺点和适用范围。

算法设计方面要设计出基于复杂度组合理论的测试场景生成算法,此算法通过组合不同的场景元素像车辆、行人、障碍物等以及参数如速度、位置等,进而生成具有不同复杂度的测试场景,之后在CARLA环境中实现并且测试该算法以验证其有效性。

仿真测试技术研究:

评价方法体系建立:建立智能驾驶系统测试评价方法体系,设计测试流程。该体系包括测试目标的设定、测试用例的选择、测试过程的执行、测试结果的分析等环节。

测试平台搭建方面要搭建仿真场景测试平台并执行大规模自动化测试,此平台能够实现对多个测试用例进行自动执行以及数据采集和结果分析从而提高测试效率和准确性



\section{创新点}

多源数据融合的场景生成方法

本研究提出融合多源数据的危险场景生成方法来融合多种数据源,该方法综合考虑车辆行驶数据、交通流量数据以及环境感知数据等,通过融合这些不同类型的数据能够更全面了解道路交通环境,进而生成更为真实且多样化的危险场景,比如结合车辆行驶数据和交通流量数据可以模拟在交通拥堵路段中车辆频繁变道、加塞等行为引发的复杂交通场景;结合环境感知数据,可以模拟出在恶劣天气条件下,车辆行驶的困难情况,如雨天路面湿滑导致的车辆失控等\cite{zhang2014multi}。

多源数据融合方法可有效提高生成场景的真实性,传统单一数据源生成方法存在数据局限性,不能全面反映道路交通环境的复杂状况,多源数据融合方法通过整合不同来源的数据,能弥补单一数据源存在的不足,所生成的场景更接近真实世界的交通情况,为自动驾驶系统的测试提供了更为可靠的依据。

基于强化学习的场景优化算法:

动态调整场景参数方面本研究设计出基于强化学习危险场景优化算法,借助智能体跟仿真环境进行交互学习来动态调整场景参数,智能体于仿真环境当中持续尝试不同场景参数组合,依据奖励函数反馈学习到最优场景参数设置,让生成的场景更具备挑战性与针对性,比如在模拟车辆超车场景的时候智能体可动态调整前车速度、加速度以及两车间距离等参数,从而生成更具危险性超车场景有效测试自动驾驶系统决策和控制能力。

强化学习算法可提高生成场景的挑战性,传统场景生成方法生成场景较简单,难以充分测试自动驾驶系统极限性能,强化学习算法通过不断优化场景参数,能生成更复杂危险场景逼迫自动驾驶系统在极端情况做决策,进而更全面评估系统性能。

自动化仿真测试框架的构建:

全流程实现自动化,本研究构建出一个完整智能驾驶危险仿真场景生成与优化框架,达成从数据采集直至仿真测试的全流程自动化。此框架涵盖数据采集与处理模块、场景生成模块、场景优化模块以及仿真测试模块等内容,各模块相互之间紧密协作配合,自动完成数据的采集、处理以及场景的生成、优化和测试等相关环节,提升了仿真测试的效率和便捷性。就像在数据采集与处理模块当中,自动从多个不同数据源采集数据并且进行预处理,于场景生成模块里面,依据预处理之后的数据自动生成危险场景,在仿真测试模块之中,自动执行仿真测试并且收集测试数据\cite{ma2018data}。

提高测试效率方面自动化仿真测试框架能显著提升效率,传统手动测试方法需耗费大量人力和时间,自动化框架可自动执行测试任务减少人工干预,进而提高测试效率,并且自动化框架还能实现大规模测试,一次性测试多个场景和参数组合获取大量测试数据,为自动驾驶系统评估和优化提供丰富数据支持。

多模态数据可视化展示技术:

本研究运用多模态数据可视化展示技术直观呈现仿真结果,把仿真结果以多种不同方式直观展现出来,不只是采用传统图表展示的方法,还借助视频展示以及3D可视化等途径,清晰直观地呈现场景与车辆行为状况,比如将仿真过程中产生的对抗性场景录制为视频,以此展示车辆于复杂交通环境里的行驶情形,又或者利用3D可视化技术,把车辆的传感器数据和环境信息以三维模型形式展示出来,让研究人员能更直观地了解车辆在仿真场景中的感知与决策过程。

多模态数据可视化展示技术可增强结果可理解性,单一图表展示难全面反映仿真结果复杂性,多模态展示方式能提供更多信息维度助研究人员全面理解,像通过视频展示研究人员可观察车辆仿真场景动态行为,了解其不同情况下反应和决策过程,通过视频展示,研究人员可以观察到车辆在仿真场景中的动态行为,了解车辆在不同情况下的反应和决策过程;通过3D可视化,研究人员可以直观地看到车辆的传感器感知范围和环境信息,更好地理解车辆的感知和决策机制。




\section{自动驾驶危险仿真场景的生成}

基于场景的仿真测试逐渐成为自动驾驶测试验证的主要手段,其中自动驾驶危险场景是仿真测试的核心和关键。危险场景定义为被测自动驾驶汽车在该具体场景完成驾驶任务的过程中可能因车辆故障、操作失效而产生碰撞风险的场景。国内外现有的自动驾驶危险场景生成方法主要有专家经验法、自然驾驶数据提取法、危险场景衍生法和基于强化学习的自生成法。


专家经验法是基于专家对自动驾驶汽车行为理解,通过行业经验、调查、判断、会议与头脑风暴等构建测试场景的方法,它最初用于低等级自动驾驶汽车测试,随着技术不断升级,也被用于制定测试标准,像ISO 34502就考虑了交通流、感知缺陷和路面特征的影响,此外专家经验法还用于自动驾驶挑战赛,通过分析事故报告设置危险场景,不过该方法存在理论性和系统性方面的缺陷,自动驾驶仿真测试的本体论提供了定义核心概念、实体、属性、关系和规则的理论框架,能帮助理解和组织测试知识体系,有学者提出融合本体论的专家经验法,构建更客观全面的场景描述与生成方法,Alexandre等人借助自动驾驶汽车的感知目标和地图信息确定相应描述本体,从而提供所有道路实体及交通参与者的概念描述,还结合专家经验法对交叉路口处的车辆运动规划功能进行验证,Chen等人提出一种三本体即交通场景、感知系统和决策算法的概念化描述,利用层次分析法实现高速路自动驾驶汽车测试场景自动生成,Jannis等人使用本体论定义自动驾驶的ODD、车辆系统架构以及V型测试模型,并用故障树分析法为自动驾驶避障算法生成50多个测试场景。Bagschik等人提出了自动驾驶系统功能和安全性分析的本体描述,利用案例研究法构建用于描述自动驾驶场景的语料库,通过自然语言处理技术来构建详细的测试场景。在专家经验法中引入本体论虽然解决了理论性、系统性方面的缺陷,但在本体构建过程中仍存在部分主观判断,同时场景维度的增多将导致其组合空间超出人类主观思维边界,使得专家经验法难以覆盖所有可能的场景类型,无法满足大规模仿真测试的需求\cite{eigen2009problem}。


自然驾驶数据提取法:自然驾驶数据提取法是指利用在现实世界采集的海量自然驾驶数据,并通过三维建模技术和虚拟仿真技术将动态元素(如车辆、行人等)进行回放以构建测试环境。典型的自然驾驶数据来源有开放道路测试数据、封闭场地数据、事故数据、路侧单元监控数据、驾驶人考试数据、驾驶模拟数据、仿真数据、标准法规测试场景数据等8种。自然驾驶数据提取法无需考虑车辆之间的交互关系,只需要对自然驾驶数据进行数据预处理、坐标系转换、3D渲染等步骤即可自动在虚拟仿真平台中实现场景的复现,具有简单、便捷的特点,被广泛地应用在百度Apollo、Waymo等自动驾驶系统开发测试过程中。夏澜等人从中国大型实车路试数据库中筛选出80例切入型危险工况样本并分类,利用聚类方法得到6类具有典型特征的危险场景,并建立了4种针对自动紧急制动系统(AEB)的测试场景。Yang 利用吉布斯抽样技术从自然驾驶数据集中采样得到356个舒适性相对较低且出现概率相对较高的测试场景片段,从而实现对于自动驾驶汽车的舒适性评估。Guo提出了一种使用随机向量场模型来建模多车相互作用的方法,应用非参数贝叶斯学习从大量自然交通数据中提取潜在的运动模式。然后,使用高斯过程来模拟多车运动,并使用狄利克雷过程创建新的危险驾驶场景。Zhao等人针对前车插入场景,采用重要度抽样方法获取自然驾驶数据集中前车插入场景的关键变量分布。实验结果表明,相对于蒙特卡罗测试,其测试效率提高了2000倍以上。Feng将高速路中行驶中的各种场景(如前车插入、换道、跟车等)进行组合,依据周车与本车位置关系的联合分布概率,查询危险驾驶场景的暴露率,并提取出暴露率高于某一阈值的场景用以构建危险场景库\cite{sayer2011integrated}。自然驾驶数据提取法具有数据量大和实现效率高的优点,但是由于轨迹数据的数字化映射不具备与测试车辆的动态交互能力,导致场景中新、旧主车的行为匹配比较困难。当测试过程中新、旧主车的行为差异比较大时,易导致仿真场景和真实场景严重错位从而导致不合理碰撞事件产生。因此,自然驾驶数据提取法多被用于小尺度算法迭代升级的研发测试。
危险场景衍生法:自然驾驶数据多在安全驾驶下收集,导致危险场景稀疏。自然驾驶数据提取法无法创造新场景,而基于深度学习的危险场景衍生技术能有效扩充危险场景数量。目前研究主要采用机器学习算法,如生成对抗网络(GAN)和变分自编码器(VAE),通过拟合大量数据分布并采样来生成新交通车辆轨迹。Krajewski等人最先提出用无监督机器学习算法训练神经网络来解决车辆轨迹建模问题,并设计了轨迹生成对抗网络和轨迹变分自动编码器两种模型。后期研究又提出一种改进的无监督BezierVAE方法,改善了轨迹在位置域和速度域的平滑度。传统的VAE方法只处理了空间上的信息,不能完全捕捉多车交互轨迹的时间特征。为此,丁文浩等人提出了一种多车轨迹生成器Multi-vehicle Trajectories Generator (MTG),它由一个双向编码器和一个多分支解码器组成,可以将多车交互场景编码为一种统一的表示,并通过采样生成新的车辆交互轨迹。针对现实世界危险驾驶场景的高维性和罕见性,Yan等人设计了一种名为NeuralNDE的深度学习生成框架用于从车辆轨迹数据中学习多智能体交互行为,并提出了一个冲突批评模型和一个安全映射网络以精细化危险场景的生成过程,使其遵循现实世界中的发生概率和模式。综上所述,应用危险场景衍生法不仅能充分挖掘真实数据的特性,还能够补充现有采集真实轨迹参数空间中缺失的部分,能够丰富场景库样本数据的多样性。但是危险场景衍生法只能够生成结构化的轨迹数据,缺乏与测试车辆的动态交互能力,在自动驾驶测试应用中具有一定的局限性。除此之外,虽然衍生数据是通过深度学习生成模型生成的,但是数据背后的真实场景特性在训练过程中会发生偏移,导致衍生轨迹难以符合车辆真实的动力学,其合理性和真实性还有待进一步的研究。\cite{eichmann2023autonomous}
基于强化学习的自生成法:为生成自然驾驶数据中难以发现的危险场景,学者开始使用强化学习来自动生成。强化学习将自动驾驶汽车与环境交互视为马尔可夫决策过程,通过定义状态、动作、奖励和状态转移函数,在仿真环境中最大化期望奖励,生成可能导致事故的危险驾驶行为,实现AI(人工智能)测试AI。Lee等人首次提出基于强化学习的自适应压力测试场景生成方法,将被测对象的测试过程建模为强化学习的训练环境,利用基于价值迭代的强化学习算法进行求解,搜索导致事故最可能的轨迹。虽然该方法主要针对航空避撞测试,但是后续也被Koren等人改进和引入到自动驾驶的避撞场景生成中,并使用蒙特卡罗树搜索(Monte Carlo Tree Search,MCTS)和前馈神经网络来寻找碰撞场景。Qin等人提出了一种交互式多智能体框架用于自动驾驶危险场景生成,将责任敏感模型作为强化学习奖励函数,引入动态约束提高场景生成的合理性,并证明了通过训练得到的背景智能体比传统的测试技术更具泛化能力。Du等人基于强化学习进行自动驾驶车辆的自适应压力测试,根据轨迹异质性设计奖励函数并引入树搜索策略,该方法生成了600多个危险场景,极大丰富了测试场景的多样性。同理,Corso等人定义了场景相异度的概念,鼓励强化学习模型发现不同类型的危险场景。
自动驾驶危险场景的参数空间一般涉及多模态情况,可大多数研究常常只关注单一模态方面,为解决这一挑战,Ding等人提出考虑反馈机制的多模态危险场景生成策略,所提强化学习算法通过加权似然最大化算法进行优化,还集成基于梯度的采样过程来提高采样效率,最终能依据学习进度调整参数区域快速发现更多危险场景,李江坤等人提出基于场景动力学和强化学习的危险场景生成方法,把随时间变化的驾驶场景用一组微分方程描述成场景动力学系统,利用神经网络作为通用函数逼近器构造场景黑盒控制器,并基于强化学习实现危险场景生成策略的求解,以超车切入场景为例,该方法在场景交互博弈、覆盖率和测试复用性等方面具备良好性能,Koren等人结合自适应压力测试和反向算法构建多精度自动驾驶危险场景生成策略,在低保真仿真器中用强化学习算法开展自适应压力测试快速构建粗粒度危险场景,随后利用反向算法将这些危险场景迁移到高保真的自动驾驶仿真器中,进而显著提升危险场景生成的效率,马依宁等人结合强化学习和遗传进化的思想提出基于不同风格驾驶模型的自动驾驶仿真测试自演绎场景生成方法,用来验证自动驾驶汽车决策结果的安全性。Feng等人提出了一种基于密集深度强化学习(Dense Reinforcement Learning,DRL)的危险场景生成策略,用以加速自动驾驶车辆安全性验证过程。所提出的密集深度强化学习可以移除非关键状态并重新连接关键状态,从而在训练数据中增加危险驾驶事件的信息密度。实验结果表明该算法可以在保证无偏性的同时显著加速自动驾驶的测试过程,比传统的真实道路测试方法快103至105倍 \cite{cheng2011analysis}。
综上所述,专家经验法因主观性逐渐退出自动驾驶仿真测试。自然驾驶数据提取法虽数据量大、效率高,但难以准确反映车辆与环境的动态交互,易导致不合理碰撞。危险场景衍生法能创造新场景,但缺乏动态交互能力,应用有限。基于强化学习的自生成法能有效生成缺少的危险场景,丰富场景类型,是未来趋势。但目前此类方法也面临以下重要难题亟待解决:1)训练产生的背景车辆过于偏激,容易构建出不可避免的碰撞场景,因而难以保证其自然性和合理性;2)倾向于生成相似、重复的危险场景,难以全面覆盖尽可能多的风险模式;3)仅针对特定道路类型和自动驾驶任务生成危险场景,难以迁移和泛化到其它交通场景。本项目将利用多目标强化学习算法自动探索考虑自然性和对抗性的危险驾驶行为生成策略,加速了自动驾驶仿真测试的效率和准确性。
