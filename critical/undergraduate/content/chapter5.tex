%!TEX root = ../../csuthesis_main.tex
\section{参考文献}

Cheng Bo, Lin Qingfeng, Song Tianjia, Cui Yongwei, Wang Limian, KuzumakiSeigo. Analysis of Driver Brake Operation in Near-Crash Situation UsingNaturalistic Driving Data[J]. International Journal of Automotive Engineering,2011,2(4).


Mckinsey \& Company. Autonomous driving’s future: Convenient and connected [EB/OL]. (2023-01-01)[2023-03-05].https://www.mckinsey.com/industries/automotive-and-assembly/our-insights/autonomous-drivings-future-convenientand-connected


ChatScene: Knowledge-Enabled Safety-Critical Scenario Generation for Autonomous Vehicles
Jiawei Zhang, Chejian Xu, Bo Li; Proceedings of the IEEE/CVF Conference on Computer Vision and Pattern Recognition (CVPR), 2024, pp. 15459-15469


张开元. FOT 数据库危险驾驶工况数据的自动处理方法研究[D]:[硕士学位论文].上海:同济大学, 2016.


Fitch G M, Hanowski R J. Using Naturalistic Driving Research to Design, Test andEvaluate Driver Assistance Systems[M]. Springer London, 2012.

朱西产, 张佳瑞, 马志雄. 安全切入场景下的驾驶人初始制动时刻分析[J]. 中国公路学报, 2019, 32(6).

杨敏明, 王雪松, 朱美新. 基于自然驾驶实验的驾驶行为研究[J]. 交通与运输,2017,33(03):7-9.

VL Neale, SG Klauer, TA Dingus, et al. The 100-Car Naturalistic Driving Study,Phase I-Experimental Design.Bahavior,2002.

TA Dingus, SG Klauer, VL Neale, et al. The 100-Car Naturalistic Driving Study,Phase II-Results of the 100-Car Field Experiment.Bahavior,2006.

Characterizing Underground Utilities[J]. Shrp Report, 2009.

Integrated Vehicle-Based Safety Systems (IVBSS) Phase I Interim Report[J]. 2008.

Sayer J, Leblanc D, Bogard S, et al. Integrated Vehicle-Based Safety Systems(IVBSS) light vehicle platform field operational test data analysis plan[J]. HeavyDuty Trucks, 2009.

Sayer J R, Leblanc D J, Bogard S E. Integrated Vehicle-Based Safety Systems(IVBSS) Third annual report[J]. Scottish Journal of Theology, 2013, 22:382-383.

Administration NHTS. Integrated Vehicle-Based Safety Systems: Light VehicleField Operational Test, Key Findings Report[J]. Annals of Emergency Medicine,2011, 58(2):205-206.

Benmimoun M, Fahrenkrog F, Zlocki A , et al. Incident detection based on vehicleCAN-data within the large scale field operational test[J].2011.

Benmimoun M, Eckstein L. Detection of Critical Driving Situations for NaturalisticDriving Studies by Means of an Automated Process[J]. bowen publishing, 2014:11-21.

Victor T, Jonas BÀrgman, Gellerman H, et al. Sweden-Michigan NaturalisticField Operational Test (SeMiFOT) Phase 1: Final Report[J]. 2010.

刘颖. 行人自动紧急制动系统测试方法研究[D]. 同济大学, 2014.

NAO Engineering, Safety \& Restraints Center, Crash Avoidance Department, “44-Crashes”, General Motors Corporation, Version 3.0, January 1997.

Crash Avoidance Metrics Partnership, “Enhanced Digital Mapping Project – FinalReport”. U.S. Department of Transportation, National Highway Traffic SafetyAdministration, November 2004.

W.G. Najm, B. Sen, J.D. Smith, and B.N. Campbell, “Analysis of Light VehicleCrashes and Pre-Crash Scenarios Based on the 2000 General Estimates System”.DOT-VNTSC-NHTSA-02-04, DOT HS 809 573, February 2003.

Najm W G, Smith J D, Yanagisawa M. Pre-Crash Scenario Typology for CrashAvoidance Research[J]. Dot Hs, 2007:767.

曾宇凡, 朱西产, 马志雄, 等. 基于真实事故和自然驾驶场景大的车辆-骑车人危险工况的统计分析[J]. Infats Proceedings of the 14th International Forum ofAutomotive Traffic Safety, 2017(01):1-9.

苏江平, 陈君毅, 王宏雁, 等. 基于中国危险工况的行人交通冲突典型场景提取与分析[J]. 交通与运输(学术版), 2017(01):216-221.

孟琳, 朱西产, 孙晓宇, 等. 真实交通危险工况下驾驶员转向避撞相关因素分析[J]. 汽车技术, 000(6):59-62.

李霖, 朱西产, 马志雄. 驾驶员在真实交通危险工况中的制动反应时间[J]. 汽车工程, 2014(10):1225-1229.

吴斌, 朱西产, 沈剑平, 等. 基于自然驾驶研究的直行追尾危险场景诱导因素分析[J]. 同济大学学报:自然科学版, 2018, 46(09):96-103.

孙 招 凤 . CAN 总 线 网 络 报 文 标 识 符 编 码 研 究 [J]. 导 弹 与 航 天 运 载 技 术 ,2009(02):34-39.

范云锋, 刘博, 郑益凯. 一种基于三次样条曲线的目标航迹拟合与插值方法研究[J]. 数字技术与应用,2019,37(03):128-129.

陈华. 面向智能辅助驾驶系统的驾驶员行为分析与建模[D].

Thiemann C, Treiber M, Kesting A. Estimating Acceleration and Lane-ChangingDynamics from Next Generation Simulation Trajectory Data[J]. TransportationResearch Record Journal of the Transportation Research Board, 2008,2088(2088):90-101.

薛薇. SPSS 统计分析方法及应用[M]. 北京:电子工业出版社, 2004.

杨学兵, 张俊. 决策树算法及其核心技术[J]. 计算机技术与发展, 2007,17(1):43-45.

曾星, 孙备, 罗武胜, 等. 基于深度传感器的坐姿检测系统[J]. 计算机科学,2018, v.45(07):243-248.

程娟, 陈先华. 基于梯度提升决策树的高速公路行程时间预测模型(英文)[J].Journal of Southeast University (English Edition),2019,35(03):393-398.79

段文强. 基于用户行为序列的网络购买行为预测[D]. 江西财经大学,2019.

Chen T, Guestrin C. XGBoost: A Scalable Tree Boosting System[J]. 2016.

师圣蔓. 基于机器学习的网络流量预测与应用研究[D]. 2019.

靳小波. 基于机器学习算法的文本分类系统[D]. 西北工业大学.

KE G, MENG Q, FINLEY T, et al. Lightgbm: A highly efficient gradient boostingdecision tree[C] // 31st Conference on Neural Information Processing Systems(NIPS 2017). California, USA,2017:3146-3154.

刘彧祺, 张智斌, 陈昊昱, et al. 基于 XGBoost 集成的可解释信用评分模型[J].数据通信, 2019(3).

Zhang Y, Xue J R, Zhang G, el al. A multi-feature fusion based traffic lightreognition algorithm for intelligent vehicles[C]. Proceeding of the 33rd ChineseControl Conference. IEEE,2014:4924-4929.

Ma C, Xue J, Liu Y, et al. Data-Driven State-Increment Statistical Model and ItsApplication in Autonomous Driving[J]. Intelligent Transportation Systems IEEETransactions on, 2018, 19(12):3872-3882.

\newpage
