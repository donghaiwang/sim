%!TEX root = ../../csuthesis_main.tex
\chapter{意图分析算法设计与实现}

\section{意图识别需求分析}

随着自动驾驶技术的不断推进,车辆对于周边环境的感知要求也日益提高。传统的目标检测与跟踪算法虽能为系统提供交通参与者的空间位置信息,但若缺乏对目标行为趋势的进一步理解,系统便难以对潜在风险做出及时响应。在复杂的城市交通环境中,车辆或行人并不总是沿规则轨迹运动,存在加速靠近、突然变道、横穿马路等高风险行为。对此,仅依赖静态边界框信息并不能满足高等级自动驾驶系统对于“先知先觉”能力的需求。因此,行为意图的分析与判别成为构建智能感知系统不可或缺的一环。

在本系统中,意图识别模块主要面向被系统持续跟踪的目标,利用其在连续帧之间的速度变化及与本车的相对距离变动情况,判断其当前运动趋势及潜在风险程度。通过在图像坐标系下计算目标中心点与视野中心之间的欧几里得距离,并结合目标自身的线速度,可实现对“靠近”“远离”“危险靠近”等行为的辨别。相比于深度学习方式构建的行为识别模型,此种基于物理建模的方式实现简单,运算代价低,适用于实时性要求较高的自动驾驶系统。同时,该方法不依赖额外训练数据,具有良好的通用性与可扩展性。

在仿真平台 Carla 提供的 Town10 与 Town01 场景中,系统通过调用车辆及传感器的同步 API 接口,在保证图像渲染实时性的同时,获取其他车辆的空间位置及动态信息。意图识别模块正是以这些基础数据为输入,构建简洁且高效的推理规则,对目标运动趋势进行实时判断。分析结果以中文文本形式叠加显示在跟踪框上,提示“目标靠近中”“目标远离中”或“危险靠近”等状态,从而构成完整的感知—识别—反馈闭环,显著提升系统对突发场景的预警能力与安全保障能力。

意图识别模块不仅补全了系统感知环节的语义层输出,同时也为后续路径规划与控制逻辑的决策提供了重要参考依据。它是连接感知与智能决策的关键桥梁,在提升系统整体智能水平方面具有重要作用。下一节将具体展开该模块的物理建模逻辑与判别策略。

\section{基于速度与距离变化的意图判别逻辑}

在自动驾驶环境中,车辆需要实时感知并理解周围目标(如其他车辆或行人)的行为趋势,以及时作出决策和控制响应。为了在视觉目标跟踪的基础上进一步提升环境感知能力,本文设计并实现了一种基于速度与距离变化的行为意图判别逻辑模块,用于实时推断跟踪目标相对于本车的动态状态,从而提供更具前瞻性的预警能力。

该模块的核心思想是:通过连续帧之间的目标相对位置变化(欧氏距离)和当前帧的目标速度,联合判断其是否存在靠近、远离或危险状态。在具体实现上,系统首先对当前帧目标的边界框进行中心点计算,结合本车视角中心作为参考点,求取目标与本车之间的距离值;随后与上一帧距离进行对比,计算两帧之间的距离变化量($\Delta d$),并结合目标当前的瞬时速度($v$)进行意图分类判断。

为提高判别的精度与稳定性,系统设置了多重判别条件,并赋予合理的速度与距离阈值,判别逻辑详见下表:

\begin{table}[htbp]
  \caption{意图识别逻辑表}
  \label{tab:timetable}
  \centering
  \begin{tabular}{ll}
    \toprule
    意图类别 & 判定条件 \\
    \midrule
    目标初始化中 & 当前为首帧,无历史距离 \\
    危险靠近 & 当前帧与本车距离$d$<150m且目标速度$v$>3.0 m/s \\
    目标靠近中 & 距离变化$\Delta d$<-5m且$v$>1.5m/s \\
    目标远离中 & 距离变化$\Delta d$>5m \\
    目标稳定 & 不满足上述任一条件 \\
    \bottomrule
  \end{tabular}
\end{table}

上述规则基于纯粹的几何物理指标进行建模,便于嵌入式部署与实时计算,同时避免了深度模型对训练样本的大量依赖。与此同时,规则判别逻辑具有良好的可解释性,便于后期维护与优化。

在系统运行过程中,判别结果会通过图形化界面实时叠加显示于目标跟踪框上,并以中文文本形式提示用户当前意图分析结论(例如“危险靠近”或“目标远离中”)。该模块与 DeepSORT 跟踪模块深度耦合,确保在单目标状态持续跟踪的同时,完成对动态意图的判别与输出。

后续工作中,可考虑将当前基于规则的模块拓展为融合规则与学习的混合模型,通过历史轨迹建模进一步提升行为预测能力。



\begin{tabular}{l l}
%  \verb|\songti| & {\songti 宋体} \\
%  \verb|\heiti| & {\heiti 黑体} \\
%   \verb|\kaiti| & {\kaiti 楷体}
\end{tabular}
